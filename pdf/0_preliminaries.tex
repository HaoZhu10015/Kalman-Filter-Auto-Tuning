\section{Preliminaries}

\subsection{Kalman Filter}

A Kalman filter mainly consists of two steps:

\underline{\bf \large Predict}

Predicted ({\it a priori}) state estimate

\begin{equation*}
    \hat{x}_{k|k-1} = F_k \hat{x}_{k-1|k-1} + B_k u_k
\end{equation*}

Predicted ({\it a priori}) estimate covariance

\begin{equation*}
    P_{k|k-1} = F_k P_{k-1|k-1} F_k^T + Q_k
\end{equation*}

\underline{\bf \large Update} 

Innovation pre-fit residual 

\begin{equation*}
    \tilde{y}_k = z_k - H_k \hat{x}_{k|k-1}
\end{equation*}

Innovation covariance 

\begin{equation*}
    S_k = H_k P_{k|k-1}H_k^T + R_k
\end{equation*}

Optimal Kalman gain 

\begin{equation*}
    K_k = P_{k|k-1}H_k^TS_k^{-1}
\end{equation*}

Updated ({\it a posteriori}) state estimate 

\begin{equation*}
    \hat{x}_{k|k} = \hat{x}_{k|k-1} + K_k \tilde{y}_k
\end{equation*}

Updated ({\it a posteriori}) estimate covariance

\begin{equation*}
    P_{k|k} = (I - K_k H_k)P_{k|k-1}
\end{equation*}

The performance of Kalman filter highly depends on the two parameters: 
\begin{itemize}
    \item State noise covariance matrix: $Q$ 
    \item Observation noise covariance matrix: $R$
\end{itemize}

Our task is to establish an auto-tuning process to find the optimal value of these two parameters under a given situation.

\textit{More details about Kalman filter see also \citep{faragher2012understanding}.}
\subsection{Objective Function}

Here we consider the Normalized Innovation Error Squared (NIS), which is computed from 

\begin{equation*}
    \epsilon_{z, k} = \tilde{y}_k^T S_{k|k-1}^{-1}\tilde{y}_k
\end{equation*}

It is often assumed that the prediction and observation errors are Gaussian. 
So when the dynamical consistency conditions are met under the chosen parameters ($Q$ and $R$), $\epsilon_{z, k}$ will be $\chi ^2$-distributed random variables with $n_z$ degrees of freedom, i.e.:

\begin{equation*}
    E[\epsilon_{z, k}] \approx n_z
\end{equation*}

where $n_z$ refers to the measurement(observation)-vector dimension.
Then an objective function for parameters tuning based on NIS can be build:

\begin{equation*}
    J_{NIS}(Q, R) = \left\lvert \log \left(\frac{\sum_{k=1}^T \epsilon_{z, k} / T}{n_z}\right)\right\rvert 
\end{equation*}

A close to zero value of this objective function will reveal a dynamical consistent estimation of the observations by the Kalman filter using the chosen parameters.

\textit{More information see also \citep{chen2018weak}, \citep{chen2019kalman}.}

\subsection{Optimization}

Here we use the Tree-structured Parzen Estimators (TPE) algorithm for optimization. 
TPE is a kind of Sequential Model-Based Optimization (SMBO) algorithm (Algorithm \ref{arg:smbo}) which iterates between fitting models and using the calculated objective function value to make choices about which parameters to investigate next.
When the searching iteration ends, SMBO will return the parameters which generate the optimal objective function value.
The main idea of TPE is similar to Bayesian optimization but the algorithm is different.
This approach can find the optimal objective function value with out stuck in the local minimum (or maximum) as well but behaves better than Bayesian optimization with Gaussian process regression.

\begin{algorithm}
    \caption{
        Sequential Model-Based Optimization (SMBO) \\
        \bm{$R$} keeps track of all target algorithm runs performed so far and their performances (objective function value), 
        $\mathcal{M}$ is SMBO's model, 
        $\bm{\vec{\varTheta }}_{new}$ is a list of promising configurations. \\
        (Modified from \citep{hutter2011sequential})
        }
    \label{arg:smbo}
    \begin{algorithmic}[1]
        \REQUIRE Target algorithm \textit{A} with parameter configuration space \bm{$\varTheta$}; instance set
        $\varPi$; objective function $J$
        \ENSURE Optimized (incumbent) parameter configuration, $\bm{\theta}_{inc}$
        \STATE $[\bm{R}, \bm{\theta}_{inc}] \leftarrow Initialize(\bm{\varTheta}, \varPi)$
        \REPEAT 
            \STATE $\mathcal{M} \leftarrow FitModel(R)$
            \STATE $\bm{\vec{\varTheta}}_{new} \leftarrow SelectConfigurations(\mathcal{M}, \bm{\theta}_{inc}, \bm{\varTheta})$
            \STATE $[\bm{R}, \bm{\theta}_{inc}] \leftarrow Intensify(\bm{\vec{\varTheta}}_{new}, \bm{\theta}_{inc}, \mathcal{M}, \bm{R}, \varPi, J)$
        \UNTIL {\it termination criteria met}
        \RETURN $\bm{\theta}_{inc}$
    \end{algorithmic}
\end{algorithm}

\textit{
    More information about SMBO: \citep{hutter2011sequential}, 
    and TPE: \citep{bergstra2011algorithms}.
}