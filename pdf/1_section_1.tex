\section{Example: Finding the optimal Kalman filter parameters for fruit fly \textit{Drosophila melanogaster} trajectory smoothing}

Suppose that we need to find the optimal Kalman filter parameters for smoothing the recorded \textit{Drosophila} 
trajectory in an arena.
We already have several recorded trajectories (observations) which are consists of the $x$ and $y$ position 
(in pixel unit) of the fly in the arena.
The basic idea is that we can first find the optimal filter parameters based on some recorded trajectories 
through the auto-tuning procedure described above, and then test the filter performance on a test trajectory.
Then the optimal parameters found can be used in the smoothing of other \textit{Drosophila} trajectories.

\subsection{Designing Kalman Filters}

Our observation of the fly position consists of a 2-dimensional vector $(x, y)$ which is recorded with a time interval of $\tau$.
Taking the velocity of fly along the two axis into account, the state vector can be represented as a 4-dimensional vector\footnote{From here we start to use $\mathbf{x}$ for state vector and $x$ for coordinate.}:

\begin{equation*}
    \mathbf{x} = [x, \dot{x}, y, \dot{y}]^T
\end{equation*}

Under the state transition function $\mathbf{x} = F\mathbf{x}$, the state transition matrix ($F$) is 

\begin{equation*}
    F = 
    \left[ 
      \begin{array}{cccc}
        1 & \tau & 0 & 0 \\
        0 & 1 & 0 & 0 \\
        0 & 0 & 1 & \tau \\
        0 & 0 & 0 & 1
      \end{array} 
    \right]
\end{equation*}

The covariance matrix of the motion noise $Q(\tau)$ is defined as a function of $\tau$:

\begin{equation*}
    Q(\tau) = 
    \left[ 
        \begin{array}{cccc}
          \frac{1}{3} \tau^3 & \frac{1}{2} \tau^2 & 0 & 0 \\
          \frac{1}{2} \tau^2 & \tau & 0 & 0 \\
          0 & 0 & \frac{1}{3} \tau^3 & \frac{1}{2} \tau^2 \\
          0 & 0 & \frac{1}{2} \tau^2 & \tau
        \end{array} 
    \right]
      \sigma^2_w
\end{equation*}

where $\sigma^2_w$ represents the variance in the motion noise.

Since the observation vector ($z$) is a 2-dimensional vector consists of the observed $x$ and $y$ coordinate of the fly, the observation matrix $H$ should be a $2 \times 4$ matrix:

\begin{equation*}
    H = 
    \left[ 
        \begin{array}{cccc}
            1 & 0 & 0 & 0 \\
            0 & 0 & 1 & 0 \\
        \end{array} 
    \right]
\end{equation*}

The observation noise covariance matrix is then defined as 

\begin{equation*}
    H = 
    \left[ 
        \begin{array}{cc}
          1 & 0 \\
          0 & 1 \\
        \end{array} 
    \right]
    \sigma^2_v
\end{equation*}

where $\sigma^2_v$ represents the variance in the observation noise.
Based on the design of motion and observation noise covariance matrix above we only need to optimize the scaler $\sigma^2_w$ and $\sigma^2_v$ later.

Besides, we also need to specify the initial state ($\mathbf{x}_0$) and state covariance matrix $P_0$:

\begin{equation*}
    \mathbf{x}_0 = [x_0, 0, y_0, 0]^T
\end{equation*}

\begin{equation*}
    P_0 = p \times 
    \left[ 
        \begin{array}{cccc}
          1 & 0 & 0 & 0 \\
          0 & 1 & 0 & 0 \\
          0 & 0 & 1 & 0 \\
          0 & 0 & 0 & 1
        \end{array} 
    \right]
\end{equation*}

where $x_0$ and $y_0$ represent the initial observation of the fly position and we set $p=1000$ which is a scaler for $P_0$.